\section{Koneksja}

{\color{red}TODO: narysować na sferce przesuwanie wektorów stycznych po fragmencie równika vs. poludnikiem do bieguna i w dół. I dodać wyjaśnienie po co są koneksje}

Niech $X$ i $Y$ będą polami wektorowymi na powierzchni $\Sigma$ imersjonowanej w $\R^3$. Definiujemy zróżniczkowanie pola $Y$ wzdłuż pola $X$ jako wektor
$$D_XY=(X(Y_1), X(Y_2), X(Y_3)),$$
który ma część styczną do $\Sigma$ oraz część normalną.




